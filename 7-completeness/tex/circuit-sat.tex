\subsection{Cook-Levin Theorem}
The theorem is often stated as:
\begin{theorem}
	Satisfiability is NP-complete.
\end{theorem}
\subsection{Circuit Satisfiability (CSAT)}
We will sketch the proof of a similar statement first:
\begin{theorem}
	Circuit satisfiability is NP-complete.
\end{theorem}

\begin{proof}
	Circuit satisfiability (CSAT) is easily seen to be in NP. A YES-instance of CSAT is a circuit whose input variables can be set to make the circuit evaluate to a 1. Given a YES-instance $c$ with $n$ input variables, a good certificate is an assignment of truth values to these input variables that cause $c$ to compute 1. A verifier only has to solve the circuit value problem that can be done in polynomial time.
	
	How do we reduce every problem in NP to CSAT? Let $B$ be any problem in NP. We will come up with a poly-time computable function $f$ mapping instances of $B$ to instances of CSAT, so that YES-instances map to YES-instances and NO-instances map to NO-instances.
	
	The key is that if we do it for any problem in NP without making assumptions about the problem, we would have shown reductions for all problems.
	
	So what do we know about any problem $B$ in NP? It has a poly-time verifier $V_B$. Remember that $V_B$ takes two inputs: an instance $x$ and a certificate $y$. The reduction $f$ that we construct will closely depend on $V_B$. $f$ must take as input an instance of $B$, say $x$, and convert it to an instance of CSAT, $f(x)$, with the answer-preserving of a reduction.
	
	Let us now examine the working of $V_B$ when dealing with instance $x$. $V_B$ takes $x$ and some certificate $y$ as inputs, and works as follows
	\begin{itemize}
		\item If $x$ is a YES-instance and $y$ is the right certificate, it accepts, i.e. it outputs 1.
		\item If $x$ is a NO-instance, for any certificate $y$, it rejects, i.e. it outputs 0.
	\end{itemize}

	Since $V_B$ runs in polynomial time, it can be converted to a poly-sized circuit with a number of inputs $= |x| + |y|$, which is polynomial in $|x|$. (Believable statement, but not fully proven here.) 
	
	Since we know the input $x$, the values for these inputs are set to match $x$. Since we do not know the input $y$, we leave the values for these inputs as Boolean variables. This is the circuit $f(x)$ which is the instance of CSAT created by our reduction. Note that the only inputs to $f(x)$ are the Boolean variables corresponding to the bits of $y$.
	
	There is a setting of the $y$ inputs that causes $f(x)$ to compute 1 if and only if $x$ is a YES-instance of B. So we have given a polynomial time reduction $f$ with the properties that 
	\begin{itemize}
		\item If $x$ is a YES-instance of B, then $f(x)$ is satisfiable.
		\item If $x$ in a NO-instance of B, then $f(x)$ is not satisfiable.
	\end{itemize}

	The reduction works from any problem $B$ in NP to CSAT. This proves that CSAT is NP-complete, and we have our first NP-complete problem.
\end{proof}

\subsection{Gates To Formulas}
\subsubsection{First Steps}
How do we write a formula (in CNF) to express that this gate is working correctly? Want to say $z = x \wedge y$ but ``='' is not a logical operator.

We can replace with $z \Leftrightarrow (x \wedge y)$, which is the same as $(z \Rightarrow (x \wedge y) \wedge ((x \wedge y) \Rightarrow z))$. Recall that $A \Rightarrow B$ is $\bar{A} \wedge B$. Using substitution, we can again rewrite as $(\bar{z} \vee (x \wedge y)) \vee (~\overline{(x \wedge y)} ~\vee z)$.

Using two rules of Boolean algebra: Distribution law for OR over AND, and De Morgan's law, we simplify to $(\bar{z} \vee x) \wedge (\bar{z} \vee y) \wedge (\bar{x} \vee \bar{y} \vee z)$.

\begin{itemize}
	\item Distribution law for OR over AND: $x \times (y + z) = (x \times y) + (x \times z)$
	\item De Morgan's law: $\overline{x \wedge y} = \bar{x} \vee \bar{y}$, $\overline{x \vee y} = \bar{x} \wedge \bar{y}$.
\end{itemize}

Check that the satisfying assignments of the last formula are all valid setting for the inputs and outputs of an AND gate. For example, if $x=0$ or $y=0$, then $z$ must be 0 to satisfy the first two clauses. But if both are 1, then $z$ must be 1 to satisfy the last clause.
\subsubsection{Using This Idea}
A similar derivation shows how to express that an OR gate is functioning correctly.

To convert a circuit into a formula, we introduce new variables on each of the internal wires of the circuit and tie the value of the output variable of a gate to the input variables using clauses as in the previous slide.



























