\documentclass[en,hazy,blue,screen,14pt]{elegantnote}
\usepackage[T1]{fontenc}
\usepackage[latin9]{inputenc}
% \usepackage[USenglish]{babel}
\usepackage{babel}
% TODO: interesting bug about the font of proof env
\usepackage{float}
\usepackage{textcomp}
\usepackage{amsmath,amsfonts,amssymb}
\usepackage{amsthm}
\usepackage{graphicx}
\usepackage[ruled,vlined]{algorithm2e}
\PassOptionsToPackage{normalem}{ulem}
\usepackage{ulem}
\usepackage{mathtools}
\usepackage{url}
\usepackage{hyperref}
\renewcommand\qedsymbol{$\blacksquare$}

% For highlight paragraph
% Usage:
%\begin{tcolorbox}[breakable,notitle,boxrule=0pt,colback=blue!20,colframe=blue!20]
%	
%\end{tcolorbox}
\usepackage{xcolor}
\usepackage{tcolorbox}
\tcbuselibrary{breakable}
\usepackage{lipsum}

\DeclarePairedDelimiter{\ceil}{\lceil}{\rceil}
\newcommand\tab[1][1cm]{\hspace*{#1}}
\newenvironment{claim}[1]{\par\noindent\underline{Claim:}\space#1}{}
\newenvironment{claimproof}[1]{\par\noindent\underline{Proof:}\space#1}{\hfill $\blacksquare$}

\title{Class Notes\\CIS 502 Analysis of Algorithm\\0-Algorithm Toolbox}
\author{Da Kuang}
\institute{University of Pennsylvania}
% \version{1.00}
\date{}

\begin{document}

\maketitle
\newpage
In this note, some intertesting discoveries and reflections are organized. Hopefully it could be used as a reference to inspire the future me.

\section{Tricks}
\subsection{Mapping Reduction}
%To explore a new algorithm, most of the time, we would like to map it or break it into some common problems. (I call them "lego" problems which will be organized in the next section.) 
When analysis a new problem $A$, we mapping reduce $A$ to a common problem $B$. Intuitively, problem $A$ is reducible to problem $B$ if an algorithm for solving problem $B$ efficiently (if it existed) could also be used as a subroutine to solve problem $A$ efficiently. We say \textbf{$A$ is reducible to $B$} or \textbf{$A \le_M B$}. 

\begin{tcolorbox}[breakable,notitle,boxrule=0pt,colback=blue!20,colframe=blue!20]

Intuitively, $A \le_M B$ means 
\begin{itemize}
	\item $A$ is not harder than $B$.
	\item $B$ is at least as hard as $A$.
\end{itemize}

\end{tcolorbox}

A reduction is a preordering, that is a reflexive and transitive relation. Note that it does not have sysmertical properity.

For example, suppose we are looking at the \textbf{closest pair of points} problem. One can argue that the \textbf{element distinct problem} can be reduced to \textbf{closest pair of points} problem. The answer of every instance of closest pair of points can be used to get the anser of an instance of element distinct problem. Therefore, 
\[\text{Element Distinct Problem} \le_M \text{Closest Pair of Points Problem} \]

So we know that closest pair of points problem is at least as hard as element distinct problem. The lower bound of element distinct problem can be used for closest pair of points problem.

\section{"Lego" Problems}
\subsection{Element Distinct Problem}
After sorting, we can discover the collision by go over the list and check if any element is same as its adjacent next element. Therefore, 
\[\text{Element Distinct Problem} \le_M \text{Sorting Problem} \]

\end{document}
