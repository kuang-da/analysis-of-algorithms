\begin{definition}
	$V$ is a \textbf{vector space over $\mathbb{R}$} if
	\begin{itemize}
		\item For $v_1, ~v_2 \in V$, $v_1 + v_2 \in V$.
		\item For any $\alpha \in \mathbb{R}, ~v \in V$, $\alpha v \in V$.
	\end{itemize}
\end{definition}

\begin{definition}
    Given a finite set of vectors, $v_1, v_2, \cdots, v_k$, then \textbf{span} 
$S$ is as 
    follows
    \[S(v_1, v_2, \cdots, v_k) = \{v: v = \alpha_1 v_1 + \alpha_2 v_2 + \cdots 
    +\alpha_k v_k, ~\alpha_i \in \mathbb{R}\}\]
\end{definition}
$S(v_1, v_2, \cdots, v_k)$ is a vector space.

\begin{definition}
	$\alpha_1 v_1 + \alpha_2 v_2 + \cdots +\alpha_k v_k$ is a \textbf{linear 
combination} of vectors. 
\end{definition}

\begin{definition}
    The set of vector $v_1, \cdots, v_k$ is called \textbf{linear dependent} if 
there exist some coefficient $\alpha_1, \cdots, \alpha_k$ not all $0$, so that 
$\sum \alpha_i v_i = 0$
\end{definition}

\begin{definition}
	A set of vector is said to be \textbf{linearly independent} if it is not 
linearly dependent.
\end{definition}

\begin{definition}
	If $v_1, \cdots, v_k$ are linearly independent and their span in $V$, then 
$v_1, \cdots, v_k$ form a \textbf{basis} of $V$.
\end{definition}

\begin{definition}
	If $v_1, \cdots, v_k$ is a basis for $v$ and $u_1, \cdots, u_m$ is another 
basis. Then $m = k$.
\end{definition}

\begin{proof}
	Suppose for contradiction that $m > k$. Since $v_i$'s from a basis,
	\begin{align*}
		u_1 &= \alpha_{11} v_1 + \cdots + \alpha_{1k}v_k\\
		u_2 &= \alpha_{21} v_1 + \cdots + \alpha_{2k}v_k\\
            &\vdots\\
        u_m &= \alpha_{m1} v_1 + \cdots + \alpha_{mk}v_k\\
	\end{align*}
Will prove $(u_1, \cdots, u_m)$ is linearly dependent. Need to show
$\sum x_i u_i = 0$ for some $(x_1, \cdots, x_m)$ not all zero.
\begin{align*}
	&x_1 u_1 + x_2 u_2 + \cdots + x_m u_m = 0\\
	\intertext{substitue $u_i$ with $v_i$'s,}
	&(x_1 \alpha_{11} + x_2 \alpha_{21} + \cdots + x_m \alpha_{m1}) v_1\\
	+~& \cdots\\
	+~&(x_1 \alpha_{1k} + x_2 \alpha_{2k} + \cdots + x_m \alpha_{mk}) v_k = 0
\end{align*}
All the coefficients above should be 0. There are $k$ coeficientes, $m$ 
equations and we assume $m > k$. Therefore, there are infinite possible 
combinations of $x_i$'s. Therefore, there must be a solution which is not 
all zeros. Because if there is not such solution, then there should be only one 
solution which is all zeros.
\end{proof}


