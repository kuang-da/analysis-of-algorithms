
Tree is a connected acyclic graph.
\subsection{Rooted tree}
\subsubsection{Inductive Definition}
A nice thing about Inductive definition is it is useful for the proofs by 
induction.
\begin{itemize}
 \item \textbf{Rule 1:} A graph consist of a single vertex v is a rooted tree 
with v as the root.
 \item \textbf{Rule 2:} If $(T_1, r_1)$, $(T_2, r_2), \cdot, (T_k, r_k)$ are 
rooted trees, then the tree $(T, r)$ consisting of a new node $r$ as root and 
edges $(r, r_1), \cdots, (r, r_k)$ is a rooted tree.
\end{itemize}
\subsection{Structural induction Proof}
Statement: Any tree with $n$ nodes has $n-1$ edges.

Since any tree can be transformed into a rooted tree, the induction can be as 
following:
\begin{itemize}
 \item Statement: Any rooted tree on $n$ nodes has $n-1$ edges.
 \item Base case: Single node tree with no edge. The statement is true.
 \item Inductive hypothesis: For a rooted tree $T_r$, built up from $(T_1, 
r_1), (T_2, r_2), \cdots, (T_n, r_n)$ using rule 2. Assume the statement is 
true for all the trees $T_1, T_2, \cdots, T_k$ and prove it for $T$.
\item Inductive step:
    \begin{itemize}
    \item Let tree $T_i$ have $n_i$ nodes, $i = 1, 2, \cdots, k$. Then $T$ has 
    $\sum_{i=1}^k n_i + 1$ nodes.
    \item By the inductive hypothesis, $T_i$ has $n_i - 1$ edges.
    \item Total number of edges is $T = \sum_{i=1}^k (n_i - 1) + k = 
\sum_{i=1}^k 
    n_i$.
    \item The number of edge is one less than the number of nodes. It proofs 
the     inductive step.
    \end{itemize}
 \end{itemize}

\paragraph{Remark:} Therefore, it takes $n-1$ comparisons to select the largest from $n$ numbers by tournament comparison.